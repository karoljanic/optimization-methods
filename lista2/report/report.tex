\documentclass{article}
\usepackage[top=3cm, bottom=3cm, left = 2cm, right = 2cm]{geometry} 
\geometry{a4paper} 
\usepackage[T1]{polski}
\usepackage[utf8]{inputenc}
\usepackage{titling}
\usepackage{caption}
\usepackage{algorithm}
\usepackage{algpseudocode}
\usepackage[parfill]{parskip}
\usepackage{multirow}
\usepackage{graphicx}
\usepackage{pgfplots}
\usepackage{stmaryrd}
\usepackage{textcomp}
\usepackage{amsmath}

\renewcommand\maketitlehooka{\null\mbox{}\vfill}
\renewcommand\maketitlehookd{\vfill\null}

\floatname{algorithm}{Algorytm}
\algrenewcommand\algorithmicrequire{\textbf{Input:}}
\algrenewcommand\algorithmicensure{\textbf{Output:}}

\title{Metody Optymalizacji}
\author{Karol Janic}
\date{1 maja 2025}

\begin{document}

\begin{titlingpage}
    \maketitle
\end{titlingpage}

\tableofcontents

\newpage

\section{Zadanie 1}
\subsection{Cel}
Celem zadania jest zaplanowanie produkcji desek w tartatu w taki sposób aby zminimalzować liczbę odpadów.
Deski mają stałą szerokośc i należy poprzecinać je w taki sposób aby zaspokoić zapotrzebowanie klientów na deski, które mogą być krótsze.

\subsection{Model}
Model parametryzowany jest szerokością desek $L \in \mathcal{R}_+$ (w calach), z której produkowane są wyroby oraz zapotrzebowaniem wyrażonym ciągiem par $(l_i, n_i)$, gdzie $1 \leq i \leq N$, $l_i$ jest szerokością deski a $n_i$ ich liczbą.

\subsubsection{Zmienne decyzyjne}
Całkowitoliczbowe zmienne decyzyjne $x_m$, gdzie $1 \leq m \leq M$ o wartościach nieujemnych określają dla każdego numeru cięcia $m$ liczbę desek pociętych w ten sposób.

\subsubsection{Funkcja celu}
Funkcją celu jest minimalizacja sumy odpadów po cięciach: $\displaystyle \sum_{m=1}^{M} x_m \cdot C_m^r$

\subsubsection{Ograniczenia}
Jedyna grupa ograniczeń wymusza spełnienie zapotrzebowania na każdą szerokośc deski:
\begin{equation*}
    \sum_{m=1}^M C_m^{l_i} \leq n_i, \qquad \qquad \qquad 1 \leq i \leq N
\end{equation*}

\subsection{Dane}
Zadana została długość deski $L = 22$ oraz zapotrzebowanie \ref{fig:zad1}.
\begin{figure}[h]
    \centering
    \begin{tabular}{l|ccc}
        $i$ & 1 & 2 & 3 \\
        \hline
        $l_i$ & 3 & 5 & 7 \\
        \hline
        $n_i$ & 80 & 120 & 110 \\
    \end{tabular}
    \caption{Zapotrzebowanie na deski}
    \label{fig:zad1}
\end{figure}

\subsection{Wyniki}
Zapisano model programowania liniowego i wyznaczono optymalne rozwiązanie dla danych. 
Wyniki przedstawiono w tabeli \ref{tab:zad1}. Taka produkcja nie generuje odpadów.

\end{document}
